
\documentclass{article}
\usepackage{amsfonts}
\usepackage{booktabs}
\usepackage{siunitx}
\usepackage{graphicx}
\usepackage{rotating} %Package added to allow the rotation of figures and chart on a page, {sidewaysfigure} command
\usepackage{tablefootnote} %Packaged added to allow footnotes in the tabular environment, use \tablefootnote command
\usepackage[round]{natbib}
\usepackage{url}
\bibliographystyle{plainnat}



\begin{document}



2). Background
2.1 Biological Factors Leading to Soil N Gradients
Nitrogen is the most common growth-limiting nutrient in temperate forests (Du et al., 2020;
LeBauer and Treseder, 2008). Nitrogen is supplied by atmospheric deposition, biological
nitrogen fixation, and in some cases geological nitrogen. These annual N inputs are usually
insufficient to meet tree growth demands, so mineralization of organic N stored in soils typically
determines site N status and forest N limitation. Whereas many temperate forest regions are
consistently N-limited, forests of the Oregon Coast Range vary naturally from N-limited to Nsaturated (Perakis and Sinkhorn, 2011). This makes the origins and consequences of such wide N
variation of special interest. Therefore, the major factors that influence soil N gradients in Coast
Range forests of the PNW are reviewed below.
2.1.1 Red Alder Mediated N-Fixation and Soil Chemistry
Biological N-fixation is an important N input to many forests. In coastal PNW forests, potential
inputs from symbiotic N-fixing red alder trees are by far the largest N source (Perakis et al.,
2011). Red alder is a hardwood tree endemic to the west coast of the United States, particularly
concentrated within the coastal Pacific Northwest (Deal and Harrington, 2006). It supports very
high rates of biological N-fixation (100-320 kg ℎ𝑎
−1 𝑦
−1
) through associations with bacteria of
the Frankia genus within nodules of its roots (Binkley et al., 1994) . Coastal red alder forests
intensify soil acidification by fostering nitrifying microbes in their associated soil (BoyleYarwood et al., 2008). Nitrification transforms ammonium into nitrate and hydrogen ions (i.e.,
leads to the release of nitric acid), allowing red alder to better access mineral pools of base
cations via nitric acid-mediated weathering of soil mineral fragments (Perakis and Pett-Ridge,
2019), while also increasing Ca leaching loss (Homann et al., 1994). In Coast Range forests with
a history of red alder growth, high N soils support a continued high annual nitrification and
nitrate leaching rates (Perakis and Sinkhorn, 2011).This results in depletion of soil exchangeable
base cations in Douglas-fir forests, with Ca depletion being most intense due to low rates of Ca
input from atmospheric deposition (Perakis et al., 2006). The legacy of red alder growth on soil
N status thus drives base cation nutrient losses in Oregon Coast Range (OCR) Douglas-fir
forests, with Ca as the most likely base cation to become limiting. 
6
2.2 Soil Parent Material, Nutrient Status, and Chemistry
OCR soils are formed over two main geologic categories of parent material; basaltic bedrock and
sedimentary bedrock. This is important to soil nutrient status as basalt is particularly cation rich
and easily weathered, whereas sedimentary bedrock is cation poor. This section gives a general
overview of the minerals that are typically found in soils formed from each parent material,
however it should be noted that soil mineralogy can be highly variable depending upon highly
specific site conditions that lead to differences in soil age.
2.2.1 Basaltic Bedrock and Soil Mineralogy
Forest soils that develop on basaltic bedrock tend to form iron oxide rich Andisols, with
kaolinite, halloysite, and gibbsite as commonly forming secondary minerals (Glasmann and
Simonson, 1985; Southard et al., 2017). These minerals have high sulfate and phosphate
adsorption capacities, limiting the mobility of these nutrients within the soil, and facilitating
anion fixation (Johnson, 1984; Pigna and Violante 2003; Inskeep, 1989). Mineral fragments
consist of olivine, pyroxenes and plagioclase (Franklin, 1990). This often produces soils that are
relatively rich in base cations. High nitrate leaching can diminish exchangeable pools of base
cations on soils derived from basaltic rock (Perakis et al., 2006). However, basaltic soils may
nevertheless contain sufficient reserves of base cations in rock fragments that are released by
mineral weathering to support plant uptake (Hynicka et al., 2016).
2.2.2 Sedimentary Bedrock and Soil Mineralogy
Parent material denoted as “sedimentary bedrock” in the OCR refers to sandstone and siltstone.
This bedrock type lends itself to the formation of andic Inceptisols rich in quartz and iron oxides,
with minor feldspar and lithic contribution to soil mineralogy (Bockheim and Langley‐
Turnbaugh, 1997; Anderson et al., 2002; Southard et al., 2017). Secondary clay mineralogy is
dominated by gibbsite and vermiculite, with small contributions of smectite in areas of
earthflow and physical disturbance (Bockheim and Langley‐Turnbaugh, 1997; Istok and
Harward, 1982). Quartz becomes increasingly dominant in soil as sedimentary sites weather
towards an Ultisol state, indicating the importance of site age on the mineral pool (Lindeburg et
al., 2013). Bulk chemistry of rock fragments in soil shows that sedimentary sites have less total
base cations than basaltic ones (Hynicka et al., 2016), and thus a potentially higher potential for
base cation nutrient depletion.
2.3 Forest Management Practices and Soil Nutrient Status
Intensive forest management is a potential, yet poorly understood, cause of nutrient loss from
PNW soils. Management practices that can influence soil nutrient depletion, such as rotation
schedules, harvest types and intensities (biomass removal), and fertilization, all vary among the
region’s forest managers. The effects of the rate of harvest and the type of harvest as they relate
to soil nutrient retention and export in the OCR are summarized in this section. Fertilization is
also reviewed; however, fertilizer blends are applied depending on soil conditions, and can vary
significantly among forest managers. 
7
2.3.1 Harvest Types and Practices
Management practices that remove different biomass pools can influence the rate of soil nutrient
depletion. This was demonstrated through calculations by Perakis et al., (2006), on effects of
stem-only and whole-tree-harvest scenarios on an OCR Douglas-fir stand. Removal of nutrientrich foliage and branches in addition to tree stems under whole-tree-harvest led to a predicted Ca
depletion time-line of 54 years. The stem-only harvest scenario, in which foliage and branches
are left on site, reduced the rate of Ca removal and increased the time to soil Ca depletion to 400
years. Different harvest practices can therefore have significant effects on site sustainability.
Incorporating different harvest types as a part of this research is critical in understanding nutrient
depletion dynamics in OCR Douglas-fir forests.
In addition to nutrient removal in biomass, forest harvesting can have indirect effects on soil
nutrient supply. Whole tree harvest combined with total forest floor and slash removal can affect
soil temperature and soil respiration, disrupt N mineralization and lead to enhanced nitrate
leaching (Fredriksen, 1971). In recently harvested forests that lack plant nutrient uptake,
increased nitrate leaching is likely to accelerate calcium leaching as well (Likens et al., 1970).
The combination of nitrate-rich, nitrifying soils of the OCR may be especially pre-disposed to
base cation limitations (Perakis et al., 2006; Perakis and Sinkhorn, 2011). Thus, whole tree
harvest might expedite the onset of Ca depletion in N saturated soils as compared to stem-only
harvest.
Nitrogen amendment via fertilization is common practice in PNW, Douglas-fir stands. Other
nutrients, such as phosphorus and potassium, are sometimes added as fertilizer as well (Miller and
Fight, 1979). The OCR is distinct from other subregions in the PNW as it has a relatively higher
N content, among the world’s top 5% most N saturated soils (Perakis and Sinkhorn, 2011). Early
N fertilizer experiments consistently show negative growth responses in OCR soils (Radwan,
1992), or more commonly frequent lack of growth response (Peterson and Hazard, 1990);
fertilization is therefore not common practice in the high N soils of the Coast Range. For low N
soils, a fertilizer regime of ~224kg/ha N as ammonium can be simulated to replicate practices.
2.3.2 Rotation Length
Increasing the rotation length of a stand allows for additional forest maturity, which can lead to
an acceleration maximization of mean annual increment and hence total wood and biomass
productivity if the stand is allowed to grow for more than 80 years (Curtis and Marshall, 1993).
Douglas-fir grown on longer rotation schedules are consistently seen to reach “optimal” woodvolume yield (Curtis, 1995). However, optimizing economic criteria such as present net value
shorten rotation length, subsequently Douglas-fir is grown on a 40-50-year schedule in order to
maximize present net value and soil expectation value (Mason, 2003; Haynes, 2005). Eighty
year rotations are more common for forests grown on public land, whereas shorter rotations with
more intensive early investments in stand establishment are currently typical on private land
(Northwest Forest Plan, Accessed May 2020). The differences between a 40 versus and 80-year
rotation on site sustainability may be significant in the long term. This effect depends on both the
age of harvest and age/size-related changes in allocation of nutrients in tree biomass components,
and ultimately on the rate of nutrient export with harvest. 
8
Forests grown on an 80-year rotation will receive half as many harvests in the span of 500 years
than those grown on a 40-year rotation. As harvest events are linked to periods of elevated nitrate
and cation leaching, shorter rotation lengths may drive nutrient depletion before longer ones.
Furthermore, 80-year (and older stands) are closer to the peak in mean annual increment (Curtis,
1995). This age is where average annual wood-volume yield is maximized.
A key consideration underlying how rotation age and forest maturity impact nutrient export is
how nutrient allocation varies with tree age. Younger trees allocate a proportionally higher
amount of nutrients to foliage, but as trees grow and stem biomass increases relative to foliar
biomass, the stem becomes an increasingly large sink of nutrients. This allocation differs by
nutrient and tissue type, as foliage contains most of the absorbed N, P, and S, whereas base
cations are more evenly distributed throughout other tissues, such as the bark and branches
(Radwan and Brix, 1986; Coons, 2014). When trees reach canopy closure, foliar biomass no
longer changes appreciably, and the trees begin to translocate mobile nutrients (N, S, P) from old
tissue to newly growing tissue, especially foliage (Miller, 1995). Immobile nutrients such as Ca
are not easily translocated from leaves, and instead are lost via litterfall, so that actively growing
tissues require uptake from soil pools of Ca (McLaughlin and Wimmer, 1999). Woody-tissue
biomass and Ca uptake continue to increase after canopy closure, and Ca uptake demand does
not necessarily decrease with tree age (Dünisch et al., 1998). The relationship between tree tissue
nutrient concentrations, age, and biomass allocation again further highlights the possibility that
Ca can be a major limiting nutrient in OCR Douglas-fir forests managed for timber production.
Nutrient allocation interacts with harvest types and harvest rates to influence nutrient depletion
rates. For example, stem-only harvest allows for a significant portion of absorbed nutrients to be
returned to the soil. Longer rotations may lead to increased total transfer of base cations from
soil to woody tissue as the stand gains biomass. However, younger stands (40-year-old) attain
woody tissue volume at a higher rate relative to mature stands (Curtis, 1995), so the rate of base
cation acquisition decreases with stand age. Repeated harvest of nutrient demanding younger
trees may accelerate the rate of nutrient removal compared to less frequent harvests of older
stands, both because of the faster rate of harvest and the faster uptake of nutrients by young trees.
2.4 Model Choice
A variety of biogeochemical models have been developed to evaluate nutrient cycling in forests,
and each model is typically developed to address specific types of questions. As the proposed
research question aims to link soil N saturation, mineral weathering, and potential nutrient
depletion from timber and/or biomass harvest, process-based models are well-suited for the
research compared to empirical models, designed to answer a different set of questions about
quantifying yield under the simplified assumption of static site quality. Process-based models are
designed to offer insights into how the internal structure of ecosystems change over time, so
parameters in this class of models dynamically interact with one-another (Korzukhin et al.,
1996). This offers an advantage over static empirically-based models, which can lack resolution
in comparing the relative weight of ecosystem processes on nutrient availability. For instance, a
model which tracks nutrient inputs and outputs based only on tree uptake and leaching with no 
9
process simulation of the interacting effects of cationic exchange would generally trace nutrient
movement, but would have no way of representing how cationic exchange itself evolves within
the system over the course of the simulation. As this study (in part) aims to identify whether or
not high nitrification rates and N saturation drive different rates of mineral weathering and
exchange site depletion, explicitly representing proton-mediated mineral weathering and
exchange site chemistry are necessary. More common models applied to forest harvest scenarios,
such as PnET-CN, CENTURY, MEL, and 3-PG tend to assume that N is limiting and therefore
do not represent base cations. These models were not designed to explicitly trace base cation
chemistry in response to N saturation over time (Metherell et al., 1993; Rastetter and Shaver,
1992; Aber et al., 1997; Sands, 2010; Rastetter et al., 2013). ORGANON and CIPSANON,
empirically based growth and yield models designed for the region, represent base cations as
nutrients, but do not represent exchange site reactivity or proton mediated mineral weathering,
and are ultimately not designed for the questions in this research (Joo et al., 2020; Hann, 2011).
Because resources were not available to develop a model to answer the questions of interest in
this research, existing models developed for other questions and that were potentially adaptable
to OCR questions were reviewed. Several forest-soil-ecosystem models were considered
regardless of classification. A table of the key characteristics of each of the main-assessed
models are described in Table 1. Models were considered applicative to the research question at
hand based on 5 characteristics:
1) Allowed for forest management scenarios to be simulated, such as tree removal, slash
removal, and applying fertilizer
2) Simulated chemical speciation, exchange site interactions, and acid-base reactions of key
nutrients
3) Linked acidity to mineral weathering (excluding mineral weathering as a mere constant input)
4) Had available code and documentation
5) Represented base cations


2.4.1 HD-Minteq
HD-Minteq is the dynamic iteration of the chemical equilibrium model Visual Minteq 3.1
combined with the PROFILE weathering model and the simple mass-balance model (Löfgren et
al., 2017). It has a robust chemical reaction simulation, with an extensive tableau listing organic
chemistry, ligand complexation chemistry, and flexible adsorption isotherms (Gustafsson, 2013).
It is also linked to tree growth and uptake models such ForSAFE and ProMod. The largest
weakness in the current model is the lack of N species transformations, in which ammonium and
nitrate concentrations are held constant, rather than having speciation calculated by a nitrification
subroutine (McGivney et al, 2019). This is of particular detriment to the current project, as the
explicit generation of acidity through nitrification is thought to be a primary driver of soil
mineral weathering in the high N soils of interest (Hynicka et al., 2016).
2.4.2 ForNBM
ForNBM uses the ModelMaker environment to mechanistically relate forest soil nutrient
availability to biomass acquisition (Cherwell Scientific Limited, 2000). Simulations of nutrient
speciation are carried by nutrient exchange equilibrium equations and a hydrogen ion module
which determines cation exchange, whereas hydrology and temperature dynamics are
represented by ForHymII (Zhu et al., 2003).
Access to ModelMaker can be made via purchase, however accessing ForNBM as developed in
ModelMaker is against the end user license agreement (Cherwell Scientific Limited, 2000).
Using the program for the purposes of this study would require defining forest-soil-atmosphere
relationships in the Pacific Northwest from “scratch”, this allows complete user flexibility in
defining process-based relationships. Yet, it also places the arduous task on the user to redefine
relationships already present in other models, and does not circumvent the need to calibrate the
model to ecosystem conditions. It is thus within the interest of time to adapt an existing processoriented model to coastal PNW conditions, rather than develop one.
2.4.3 FORECAST-ForWaDy
This model connects the hybrid growth-yield model of FORECAST with a process-based
hydrology model ForWaDy. FORECAST simulates tree growth based on site nutrient
availability and light, while ForWaDy incorporates hydrology and temperature effects on soil
moisture and litter decomposition (Seely et al., 2015). Nutrient availability in FORECAST is
simulated through a mass-balance approach, essentially balancing ecosystem outputs and inputs,
but not simulating chemical speciation explicitly (Seely et al., 2010). Instead, nutrient content in
excess of the defined CEC and uptake requirements are leached out from the soil (Kimmins et
al., 1999). The casual relationship between soil N saturation, acidity, and mineral weathering
would thus be lost. FORECAST-ForWaDy models the effects of climate change on soil moisture 
11
and water dynamics effectively, however the treatment of soil chemistry is not explicit, and
therefore is inadequate for this research.
2.4.3 PnET-BGC
This model is the conjoined product of the PnET-CN and a biogeochemical cycling (BGC)
model. It is comparable to the Nutrient Cycling Model in that it simulates long-term forest
processes such as cation exchange, anion adsorption, water movement, uniform forest growth,
and mineral weathering inputs (Gbondo‐Tugbawa et al., 2001). The model requires a spin-up
phase, in which parameters are allowed to reach a steady state in the model before the testing
phase begins, based on hindcast predictions of historical climate (Driscoll, Accessed May 2020).
Pnet-BGC considers soil as a single layer, and thus may neglect to adequately represent depthrelated differences in physical and chemical properties of soil that occur with depth. A two-layer
version of the model was developed and tested; however current documentation suggests that
this is not a common part of the standard model available for use (Chen and Driscoll, 2005;
Driscoll, Accessed May 2020). This may lead the model to be less robust in its simulations of
hydrology, and can neglect to capture changes in mineral content, exchange site reactivity, and
other belowground processes that vary with depth.
2.4.4 NutsFor
The NutsFor model is a dynamic forest-process model developed by van der Heijden et al. 2017.
It is a hybrid model that uses documentation from the Nutrient Cycling Model (NuCM), the
ForSAFE model, the PROFILE mineral weathering model, and the WATFor forest hydrology
model (van der Heijden et al., 2017; Sverdrup and Warfvinge, 1988; Legout et al., 2016). It was
developed during research at the Breuil-Chenue site, located in Burgundy, France.
As most of the NutsFor documentation concerning soil chemistry comes from NuCM, the
development of NuCM is further discussed. NuCM was developed by the Electric Power
Research Institute in 1986 with Tetra Tech, INC, as part of the Integrated Forest Study
(Kvindesland, 1997). The model was developed to simulate the effects of atmospheric acidic
deposition on forest nutrient cycling, and thus simulates acid-generating and depositional
processes explicitly. NuCM was originally parameterized and calibrated to the Smokies Tower,
Tennessee, and Huntington Forest, New York, sites (Munson et al., 1992). It has been applied to
a wide variety of forests and forest scenarios (Johnson et al., 2000), specifically to the PNW and
Douglas-fir (Verburg et al., 2001). NuCM is characteristically flexible in its application, seeing
usage in temperature forests in China (Yu et al., 2008), and chaparral forests in California (Fenn
et al., 1996). The latter application to the dry, N saturated forest-soil in California demonstrated
NuCM’s ability to both simulate drought-like conditions and predict N saturation. These
conditions are highly relevant to the OCR’s summer-time precipitation regime and soil N status.
NuCM also has a history of being utilized to simulate the long-term effects of crop rotation,
repeated harvest, and soil acidification on soil nutrient cycling (Liu et al., 1991; van der Heijden
et al., 2011; Verburg et al., 2001).
Of the process-based, forest-soil-atmosphere models, NutsFor captures the majority of
processes relevant to forest nutrient cycling. As NuCM was developed with rather simple 
12
hydrology and mineral weathering sub-models, the integration of PROFILE and WATFor with
NuCM makes NutFor the ideal model to simulate nutrient cycling dynamics in the OCR.
3). Methods
3.1 Model Implementation
As the methodology of model experimentation is highly specific to model structure, it is
necessary to further introduce model design, terminology, and hypothesis testing within the
context of the selected model, NutsFor. The structure of NutsFor is shown conceptually in Figure
1. Model parameterization and model calibration are often used interchangeably; however,
within the framework of NutsFor, parameterization is the process of selecting representative
values for sites, whereas calibration is the process of adjusting current model outputs to increase
the utility of the model for different conditions. Failed calibration can lead to parameterization
when parameters are not appropriate for a site. Calibration and use of NutsFor in this research
will be done in collaboration with its developer, Dr. Gregory van der Heijden (personal
communication, May 2020).
3.1.1 Methods of Parameterization
Model parameterization will largely be completed using datasets from OCR literature (Table
A1), regional databases, and through calibration. Unknown parameters can be entered based on
estimation, empirically driven relationships, related datasets, or through field measurements.
Figure 1: NutsFor model design, displaying the major nutrient cycling processes represented
in NutsFor. Atmospheric Deposition includes wet, dry, and cloud interception, nutrient
cycling includes litterfall, foliar exudation/leaching, and interception storage of deposited
particles.
13
However as with all models, uncertainty in parameterization can lead to uncertainty of obtained
results.
Data on stem, root, and branch chemistry will be generated from representative sites from field
samples across coastal soil N and mineralogical gradients. This will include stem sapwood and
heartwood samples, branch samples, and root samples. These data can be combined with
approximations of tree biomass allocation to generate a more accurate stoichiometry for
simulated trees, as datasets which include Douglas-fir stoichiometries are either incomplete with
respect to NutsFor parameters (i.e., they don’t include Cl, Na, Al, for all corresponding biomass
compartments) or are measured in stands that do not grow along the proposed N and
mineralogical gradients.
3.1.2 Model Calibration
NutsFor will be calibrated following NuCM user manual recommendations (Munson et al.,
1992). The process entails using lysimeter or stream ion data to constrain simulated ion and
water flow to real world parameters. These constraints ensure that aberrant or otherwise
unreasonable calculations are avoided, allowing the user to shape soil properties, such as nutrient
mobility and hydrology, for a given site.
3.1.3 Experimental and Site Design
Hypothesis 1:
“Highly N saturated soils will drive cationic leaching independent of site mineralogy.”
Hypothesis one will be tested by simulating high and low N sites according to the generally
observed N and mineral status of each site. High N sites are defined as those at or exceeding
0.4% soil N (0-10cm depth), which is associated with threshold increases in relative nitrification
up to 100% and dominance of nitrate leaching in the rooting zone (Perakis and Sinkhorn, 2011).
These characteristics can be used to parameterize each site according to the proposed gradient.
The results of NutsFor model output will be interpreted as supporting hypothesis 1 if the
simulated rates of base cation leaching are more closely related to metrics of N-saturation (e.g.,
nitrification, nitrate leaching) than with the base cation content of minerals in contrasting basaltic
vs sedimentary bedrock. I expect that highly N-saturated and nitrifying soils will drive exchange
site cation depletion in NutsFor and will increase soil mineral weathering rates, however the
exchange site depletion may limit growth before mineral weathering rates can reach adequate
levels to replenish the exchange complex.
Hypothesis 2:
“More intensive management scenarios (whole tree harvest and shorter rotations) will increase
the rate of nutrient depletion compared to less intensive management scenarios (stem-only
harvest and longer rotations).”
Hypothesis 2 requires that the symptoms of nutrient depletion are defined. NutsFor can show
nutrient depletion and nutrient limitation in several ways. One method would be to compare
foliar chemistry to established values held to be indicative of nutrient limitation (van den 
14
Driessche, 1979). These values can then be compared to soil solution and exchange site
chemistry, which is informative of the soil’s ability to supply nutrients to the ecosystem in the
short term. Finally, the best indicator of nutrient depletion would be biomass accrual rate and
total amount at the time of harvest. As nutrient depletion occurs, NutsFor will simulate a slower
growth rate as compared to the “idealized” growth curve (Figure 2). This occurs when uptake
demand of any nutrient (set to defined values) exceeds soil solution and exchange site
concentrations. The results will be said to be in support of hypothesis 2 if biomass acquisition of
Douglas-fir under more intensive management scenarios fall to unviable levels before (after
fewer rotations) those under less intensive scenarios. I expect that the more intensive
management scenarios (40-year rotations, WTH) with the most mineral poor soil (sedimentary)
and highest base cation leaching (high N), will become depleted after fewer rotations than less
intensive scenarios, lower N soils, and cation-rich basaltic bedrock soils. The longest lasting
stands in terms of Ca depletion are expected to be the low N, basaltic, 80-year rotations under
stem-only harvest, though it is possible these stands may exhibit signs of N deficiency as well.
The harvest scenarios and site characteristics are described in Table 2. In total, there are five
harvest scenarios and four sites, for a total of twenty model-runs.
NuCM will be independently calibrated and divided into 4 representative sites, based on soil N
status and parent material.
1. Low N, Sedimentary
2. High N, Sedimentary
3. Low N, Basaltic
4. High N, Basaltic
Each site may then undergo differential harvest events, stem only or whole-tree-harvest and 40-
year or 80-year rotation. Every site will also be allowed to run towards an old growth (~500
years) state, allowing for long-term site development of soil chemical characteristics to be
examined in absence of harvest removals.
Table 2:

Hypothesis 3:
Hypothesis 3a: “The rate of nutrient depletion in sedimentary sites will be greater than that in
basaltic sites because the weathering capacity of minerals in basaltic sites will provide larger
mineral weathering base cation inputs.”
Hypothesis 3b: “Mineralogical differences among bedrock-type and consequent dynamics of Ca
availability through weathering have a larger effect than static estimates of the exchangeable Ca
pool on the long-term sustainability of forest Ca supply, as assessed by tree growth over time.”
Hypothesis 3 will be tested by specifying the mineral pools of the sedimentary and basaltic sites.
The mineral pool of sedimentary sites will be set according to observed results from Anderson et
al., (2002). The mineral pool for the basaltic sites will be defined based on known mineralogy of
basaltic soils, known content of rock fragments, and based on whole-rock chemistry for these
soils (Franklin, 1990, Hyncika et al., 2016). As the capacity of the mineral pool to supply base
cations via weathering is dependent on the weathering state and age of a given soil, mineralogy
Figure 2: A representation of an idealized biomass curve (fit
logarithmically) for Oregon Coast Douglas-fir (black), plotted with the
curve of a nutrient limited Douglas-fir stand (red).
0
200
400
600
800
1000
1200
0 100 200 300 400 500
Total Aboveground Biomass (Mg/ha)
Stand Age (yr)
Log. (Idealized) Log. (10% Reduction)
16
and soil chemical data will be selected from sites within similar age ranges and weathering
states.
The results of the NutsFor model output will be interpreted as supporting hypothesis 3a if
sedimentary sites are observed growing below the target biomass curve after fewer rotations than
basaltic sites, and the weathering input of base cations of basaltic sites is consistently higher than
that simulated in sedimentary sites. Hypothesis 3b will be similarly interpreted by comparing
biomass acquisition between sedimentary and basaltic sites. It is expected that the weathering
inputs of the basaltic sites will be greater than that of the sedimentary sites, and that weathering
fluxes will be elevated in high N soils as compared to low N soils. Due to this, it is further
expected that basaltic sites will be able to supply base cations to the exchangeable complex and
stimulate growth at a faster rate than sedimentary sites.
A summary of expected results is shown in Figure 3.
3.1.4 Harvest Simulation
Simulating harvest events entails setting tree age down from the age of harvest to year 1, at
which time the model can be reparameterized with resultant output data of the previous rotation,
and allowed to run again for another rotation. This process would occur until expected tree
growth falls to unviable ranges (indicating soil depletion of a nutrient), or to 500 years, in the
case where growth never becomes limited.
Figure 3: Expected biomass curves for sites, without the consideration of
harvest. Low N scenarios are expected to have similar growth curves, but are
separated to show the lines clearly.

The post-harvest simulation stage illustrated in Figure 4 will be conducted between rotation
simulations. It is a small 3-year simulation phase which will be used to simulate both postharvest soil characteristics (such as temperature increases) and the addition of nutrients after
fertilization, from slash left over after harvest and perhaps other silvicultural or operational
activities. These simulations will require modification of the NutsFor model to represent biomass
removal and transfers in whole tree harvest versus stem only harvest scenarios.
Figure 4: Depiction of model run structure. End of model runs will occur after observed
depletion or after 500 years of simulation.
3.4 Analysis of Results
Once a stand exhibits the symptoms of nutrient depletion, such as a reduction in biomass accrual
rate and chronic net export of key nutrients, simulations can be discontinued and results can be
collected from this final stage. The results can be further compared to the old growth behavior,
which acts as a control scenario in which harvest is not a driving variable on nutrient depletion,
but initial soil N status and soil chemistry influence long-term nutrient accumulation and
availability in absence of disturbance.
3.4.1 Examination of Nutrient Depletion
As identified previously, biomass acquisition over time will be the key metric by which
depletion is determined. This will be achieved by comparing NutsFor simulated total
aboveground biomass to an idealized biomass curve of Douglas-fir. If the simulated biomass is
severely below the biomass of a typical, productive Douglas-fir stand of the same age, then the
soil can be said to be nutrient limited. Further analyses can be conducted to identify the
particular limiting nutrient. Data analysis will further include comparing the rates of mineral
weathering to rates of change in exchangeable base cations and tree uptake of base cations,
indicating whether or not mineral weathering rates match exchange site depletion due to uptake
and proton-driven leaching.
18
NutsFor simulates nutrient limitation by checking the soil solution and nutrients adsorbed in the
exchangeable pool against uptake demands. Deficiencies of any nutrient will then lead to
reduced uptake of all nutrients by some defined percent, and thus a reduction in growth rate.
Reduction of growth rate will occur when a limiting nutrient reaches 80% of the availability
required for tree growth, this will in turn drive the model to uptake 80% of all required nutrients
(Munson et al., 1992). This means that further analysis must be taken to identify the particular
limiting nutrient, as all nutrients will be absorbed at a lesser amount during limitation. By
calculating input-output budgets of nutrients in the stands after the incidence of depletion, the
depleted nutrient can be identified as that which is lost from the ecosystem due to harvest and
leaching, at the highest rate relative to the uptake demands of the stand (van der Heidjen et al.,
2017). There may be co-limiting nutrients at any given time point, however co-limitations can be
identified in a similar manner.
3.4.2 NutsFor Sensitivity to Nitrification
A sensitivity analysis can be conducted to identify the role of nitric acid as a driver of mineral
weathering and cation exchange depletion. This will be conducted in a manner similar to that
suggested by Munson et al., (1992), in which nitrification rates are set to the lowest observed
value, and incrementally increased until the highest observed nitrification rate is reached (Perakis
and Sinkhorn, 2011). Nitrification rates will be increased in regular increments, and leaching
rates and mineral weathering rates will be monitored simultaneously. For this analysis, all other
variables will be held constant, including management scenarios. This is a necessary restriction,
as harvest events drive soil processes which enhance nitrification, and can confound the explicit
relationship between defined nitrification rates and weathering rates.
The subsequent change in mineral weathering rates and exchange site depletion can then be
compared across sites for every increment of nitrification, revealing the level of nitrification that
is necessary for cation nutrient depletion to occur. As nitrification occurs at a soil N threshold,
the sensitivity analysis may show a similar relationship as shown in Figure 5, in which mineral
weathering increases after biotic demand of nitrate is surpassed by nitrate production within the
soil. Mineral weathering can then increase as nitrification further increases. The supply in
exchangeable Ca may follow a pattern similar to Figure 6, in which Ca is leached from the
exchangeable pool as nitrate concentrations increase, similar to threshold behavioral trends
observed in Perakis et al., (2006).




\bibliography{Manuscript}


\end{document}