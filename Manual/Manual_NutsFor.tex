% Options for packages loaded elsewhere
\PassOptionsToPackage{unicode}{hyperref}
\PassOptionsToPackage{hyphens}{url}
%
\documentclass[
]{article}
\usepackage{lmodern}
\usepackage{amssymb,amsmath}
\usepackage{ifxetex,ifluatex}
\ifnum 0\ifxetex 1\fi\ifluatex 1\fi=0 % if pdftex
  \usepackage[T1]{fontenc}
  \usepackage[utf8]{inputenc}
  \usepackage{textcomp} % provide euro and other symbols
\else % if luatex or xetex
  \usepackage{unicode-math}
  \defaultfontfeatures{Scale=MatchLowercase}
  \defaultfontfeatures[\rmfamily]{Ligatures=TeX,Scale=1}
\fi
% Use upquote if available, for straight quotes in verbatim environments
\IfFileExists{upquote.sty}{\usepackage{upquote}}{}
\IfFileExists{microtype.sty}{% use microtype if available
  \usepackage[]{microtype}
  \UseMicrotypeSet[protrusion]{basicmath} % disable protrusion for tt fonts
}{}
\makeatletter
\@ifundefined{KOMAClassName}{% if non-KOMA class
  \IfFileExists{parskip.sty}{%
    \usepackage{parskip}
  }{% else
    \setlength{\parindent}{0pt}
    \setlength{\parskip}{6pt plus 2pt minus 1pt}}
}{% if KOMA class
  \KOMAoptions{parskip=half}}
\makeatother
\usepackage{xcolor}
\IfFileExists{xurl.sty}{\usepackage{xurl}}{} % add URL line breaks if available
\IfFileExists{bookmark.sty}{\usepackage{bookmark}}{\usepackage{hyperref}}
\hypersetup{
  pdftitle={Nutrient Cycling in Forest Ecosystems: User Manual v.()},
  pdfauthor={Kaveh Gholamhossein Siah \& Gregory van Der Heijden},
  hidelinks,
  pdfcreator={LaTeX via pandoc}}
\urlstyle{same} % disable monospaced font for URLs
\usepackage[margin=2 cm]{geometry}
\usepackage{color}
\usepackage{fancyvrb}
\newcommand{\VerbBar}{|}
\newcommand{\VERB}{\Verb[commandchars=\\\{\}]}
\DefineVerbatimEnvironment{Highlighting}{Verbatim}{commandchars=\\\{\}}
% Add ',fontsize=\small' for more characters per line
\usepackage{framed}
\definecolor{shadecolor}{RGB}{248,248,248}
\newenvironment{Shaded}{\begin{snugshade}}{\end{snugshade}}
\newcommand{\AlertTok}[1]{\textcolor[rgb]{0.94,0.16,0.16}{#1}}
\newcommand{\AnnotationTok}[1]{\textcolor[rgb]{0.56,0.35,0.01}{\textbf{\textit{#1}}}}
\newcommand{\AttributeTok}[1]{\textcolor[rgb]{0.77,0.63,0.00}{#1}}
\newcommand{\BaseNTok}[1]{\textcolor[rgb]{0.00,0.00,0.81}{#1}}
\newcommand{\BuiltInTok}[1]{#1}
\newcommand{\CharTok}[1]{\textcolor[rgb]{0.31,0.60,0.02}{#1}}
\newcommand{\CommentTok}[1]{\textcolor[rgb]{0.56,0.35,0.01}{\textit{#1}}}
\newcommand{\CommentVarTok}[1]{\textcolor[rgb]{0.56,0.35,0.01}{\textbf{\textit{#1}}}}
\newcommand{\ConstantTok}[1]{\textcolor[rgb]{0.00,0.00,0.00}{#1}}
\newcommand{\ControlFlowTok}[1]{\textcolor[rgb]{0.13,0.29,0.53}{\textbf{#1}}}
\newcommand{\DataTypeTok}[1]{\textcolor[rgb]{0.13,0.29,0.53}{#1}}
\newcommand{\DecValTok}[1]{\textcolor[rgb]{0.00,0.00,0.81}{#1}}
\newcommand{\DocumentationTok}[1]{\textcolor[rgb]{0.56,0.35,0.01}{\textbf{\textit{#1}}}}
\newcommand{\ErrorTok}[1]{\textcolor[rgb]{0.64,0.00,0.00}{\textbf{#1}}}
\newcommand{\ExtensionTok}[1]{#1}
\newcommand{\FloatTok}[1]{\textcolor[rgb]{0.00,0.00,0.81}{#1}}
\newcommand{\FunctionTok}[1]{\textcolor[rgb]{0.00,0.00,0.00}{#1}}
\newcommand{\ImportTok}[1]{#1}
\newcommand{\InformationTok}[1]{\textcolor[rgb]{0.56,0.35,0.01}{\textbf{\textit{#1}}}}
\newcommand{\KeywordTok}[1]{\textcolor[rgb]{0.13,0.29,0.53}{\textbf{#1}}}
\newcommand{\NormalTok}[1]{#1}
\newcommand{\OperatorTok}[1]{\textcolor[rgb]{0.81,0.36,0.00}{\textbf{#1}}}
\newcommand{\OtherTok}[1]{\textcolor[rgb]{0.56,0.35,0.01}{#1}}
\newcommand{\PreprocessorTok}[1]{\textcolor[rgb]{0.56,0.35,0.01}{\textit{#1}}}
\newcommand{\RegionMarkerTok}[1]{#1}
\newcommand{\SpecialCharTok}[1]{\textcolor[rgb]{0.00,0.00,0.00}{#1}}
\newcommand{\SpecialStringTok}[1]{\textcolor[rgb]{0.31,0.60,0.02}{#1}}
\newcommand{\StringTok}[1]{\textcolor[rgb]{0.31,0.60,0.02}{#1}}
\newcommand{\VariableTok}[1]{\textcolor[rgb]{0.00,0.00,0.00}{#1}}
\newcommand{\VerbatimStringTok}[1]{\textcolor[rgb]{0.31,0.60,0.02}{#1}}
\newcommand{\WarningTok}[1]{\textcolor[rgb]{0.56,0.35,0.01}{\textbf{\textit{#1}}}}
\usepackage{longtable,booktabs}
% Correct order of tables after \paragraph or \subparagraph
\usepackage{etoolbox}
\makeatletter
\patchcmd\longtable{\par}{\if@noskipsec\mbox{}\fi\par}{}{}
\makeatother
% Allow footnotes in longtable head/foot
\IfFileExists{footnotehyper.sty}{\usepackage{footnotehyper}}{\usepackage{footnote}}
\makesavenoteenv{longtable}
\usepackage{graphicx}
\makeatletter
\def\maxwidth{\ifdim\Gin@nat@width>\linewidth\linewidth\else\Gin@nat@width\fi}
\def\maxheight{\ifdim\Gin@nat@height>\textheight\textheight\else\Gin@nat@height\fi}
\makeatother
% Scale images if necessary, so that they will not overflow the page
% margins by default, and it is still possible to overwrite the defaults
% using explicit options in \includegraphics[width, height, ...]{}
\setkeys{Gin}{width=\maxwidth,height=\maxheight,keepaspectratio}
% Set default figure placement to htbp
\makeatletter
\def\fps@figure{htbp}
\makeatother
\setlength{\emergencystretch}{3em} % prevent overfull lines
\providecommand{\tightlist}{%
  \setlength{\itemsep}{0pt}\setlength{\parskip}{0pt}}
\setcounter{secnumdepth}{-\maxdimen} % remove section numbering
\usepackage{subfig}
\usepackage{booktabs}
\usepackage{longtable}
\usepackage{array}
\usepackage{multirow}
\usepackage{wrapfig}
\usepackage{float}
\usepackage{colortbl}
\usepackage{pdflscape}
\usepackage{tabu}
\usepackage{threeparttable}
\usepackage{threeparttablex}
\usepackage[normalem]{ulem}
\usepackage{makecell}
\usepackage{xcolor}
\ifluatex
  \usepackage{selnolig}  % disable illegal ligatures
\fi

\title{Nutrient Cycling in Forest Ecosystems: User Manual v.()}
\author{Kaveh Gholamhossein Siah \& Gregory van Der Heijden}
\date{05 November 2020}

\begin{document}
\maketitle

{
\setcounter{tocdepth}{3}
\tableofcontents
}
\hypertarget{introduction}{%
\subsubsection{Introduction}\label{introduction}}

The Nutrient Cycling in Forested Ecosystems (NutsFor) model is a
stand-level biogeochemical model which simulates the forest nutrient
dynamics of 16 chemical species These processes include hydrology, soil
exchange site chemistry, mineral weathering, stand growth and uptake,
and atmospheric deposition. This manual describes the most recent
iteration of NutsFor (version \ldots\ldots), which further includes
stand harvest, biomass residues, separate pools of soil water, and other
small changes made to the original model described in van der Heijden et
al, 2017. All changes from the original model are noted with an asterisk
(``*''). As NutsFor is an active project of Gregory van der Heijden,
model capabilities are open to change based on collaboration.

This user manual is intended to give relevant information about the
model's capabilities, useful tips regarding its usage, and insight into
the equations and theory which link model parameters. A list of
equations used to parameterize NutsFor are also included, along with
some description of how these equations are used.

\hypertarget{model-start-up}{%
\subsubsection{Model Start Up}\label{model-start-up}}

Prior to running or implementing change to model parameters, the user
should go through the input and output files for NutsFor using excel.
NutsFor does not currently have a built-in method to visualize results,
it is recommended that the user use excel macro commands or R to
automatically read generated files (which will always output with the
same structure) to visualize results. This is highly important in the
calibration phase of model implementation.

The model is simple to run, simply double-click or other wise enact the
NutsFor.exe file. The model will run and output results to its
corresponding .csv files automatically. If the output files are moved
from the NutsFor directory or removed, NutsFor will generate new csv
files as long as the Output folder is still present in the NutsFor.exe
directory. A successful model run will be accommodated by a ``Exporting
Model Run Data'' and ``Model Run End-Press Return to Close''.
Unsuccessful model runs (i.e., crashes) will result in either the
NutsFor exe closing on its own or on the former messages never appear in
the NutsFor.exe box.

\hypertarget{input-files}{%
\subsubsection{Input Files}\label{input-files}}

There are 10 input files for NutsFor (the .nut files). These files are
csv files which can be opened through excel. A full list of the input
parameters, separated by the file they appear in, is shown in the tables
below. Below each table is a more extensive description of each variable
and its effects on model outputs.

\hypertarget{dry-deposition}{%
\paragraph{Dry Deposition}\label{dry-deposition}}

\begin{table}

\caption{\label{tab:unnamed-chunk-3}Dry Deposition File}
\centering
\resizebox{\linewidth}{!}{
\begin{tabular}[t]{>{}l>{\raggedright\arraybackslash}p{10cm}l}
\toprule
Input & Units & Description\\
\midrule
\textbf{Ca, Mg, K, Na, NO3, NH4, SO4, Cl, PO4, DOC, H, Al, Si} & $\mu$mol/L & Determines the amount of each species deposited into ecosystem by Dry Deposition\\
\bottomrule
\end{tabular}}
\end{table}

*Description of Parameters: Dry Deposition

The dry deposition file is comparable to the Rain Concentration file. It
simply delegates the deposition of chemical species without the
deposition of water. The file starts with An and Mois (Year and Month
Columns), and then lists out chemical species from left to right.

\hypertarget{soil-parameters}{%
\paragraph{Soil Parameters}\label{soil-parameters}}

\begin{table}[!h]

\caption{\label{tab:unnamed-chunk-4}Soil Parameters File}
\centering
\resizebox{\linewidth}{!}{
\begin{tabular}[t]{lll}
\toprule
Input Parameter & Units & Description\\
\midrule
\cellcolor{gray!6}{Soil Layer} & \cellcolor{gray!6}{N/A} & \cellcolor{gray!6}{Desingates number of layers.}\\
Soil Layer & N/A & Assigns layer number to each layer\\
\cellcolor{gray!6}{Depth} & \cellcolor{gray!6}{cm} & \cellcolor{gray!6}{Tracks vertical depth of soil layer}\\
Thickness & cm & Designates the individual thickness of each soil layer\\
\cellcolor{gray!6}{Bulk Density} & \cellcolor{gray!6}{$kg/m^3$} & \cellcolor{gray!6}{N/A}\\
\addlinespace
Solid Density & $kg/m^3$ & N/A\\
\cellcolor{gray!6}{Stoniness} & \cellcolor{gray!6}{Percent} & \cellcolor{gray!6}{Determines the percent of soil particles greater than 2mm in diameter}\\
Root Fraction & Fraction & Determines fraction of total tree root biomass in eah layer\\
\cellcolor{gray!6}{pKGibb} & \cellcolor{gray!6}{N/A} & \cellcolor{gray!6}{Gibbsite formation constant}\\
pkAl precip & N/A & Al precipitation constant (unused in current version)\\
\addlinespace
\cellcolor{gray!6}{pCO2} & \cellcolor{gray!6}{Unknown} & \cellcolor{gray!6}{Partial Pressure of CO2}\\
Area & $m^2/m^3$ & Weatherable mineral area\\
\cellcolor{gray!6}{Wilting Point} & \cellcolor{gray!6}{Percent} & \cellcolor{gray!6}{Soil moisture at which stand cannot extract water}\\
Field Capacity & Percent & Soil moisture after excessive water is leached\\
\cellcolor{gray!6}{Saturation} & \cellcolor{gray!6}{Percent} & \cellcolor{gray!6}{Soil moisture at which all pore space is filled with water}\\
\addlinespace
CEC & $c{mol}_c/kg$ & Total cation exchange capacity\\
\cellcolor{gray!6}{KH Cation} & \cellcolor{gray!6}{N/A} & \cellcolor{gray!6}{Gapon coefficients for each cation}\\
Percent Mineral Area & Percent & Porportion of total mineral area that each individually defined mineral occupies\\
\cellcolor{gray!6}{Uptake Fraction Per Layer (all nutrients)} & \cellcolor{gray!6}{Fraction (max 1)} & \cellcolor{gray!6}{Determines fraction of needed nutrient which stand uptakes from each}\\
Nitrification (ks, knit) & ks: $\mu$mol/L, knit: Fraction & Nitrification constants that determines NO3 production\\
\addlinespace
\cellcolor{gray!6}{Microbial Turnover} & \cellcolor{gray!6}{Multiple of Microbial Pool} & \cellcolor{gray!6}{Determines fraction of SOM which turns over}\\
Organic Matter Parameters & g Nutrient/kg soil) & Iniital Amount of Nutrient Mineralized in the SOM pool.\\
\cellcolor{gray!6}{Soil Solution Initial (Multiple Species)} & \cellcolor{gray!6}{$\mu$mol/L} & \cellcolor{gray!6}{Concentration of all chemical species at beginning of simulation}\\
Soil Adsorbed Anions Initial & $\mu$mol/kg & Adsorbed Concentration of anions at the beginning of the simulation\\
\cellcolor{gray!6}{CEC Initial} & \cellcolor{gray!6}{c${mol}_c$/kg} & \cellcolor{gray!6}{Sorbed Concentration of Cations to the CEC at the beginning of the simulation}\\
\bottomrule
\end{tabular}}
\end{table}

Description of Parameters: Soil Parameters

\textbf{\emph{Soil Layers:}}

NutsFor allows any number of soil layers to be set in the model. Extra
layers are added by typing in the new layer quantity into the
Soil\_Layer parameter, and adding the new layer to the Soil Parameter
file by parameterizing the new layer as is done for layers before it.
When a new layer is added, the spacing between the different tables in
the soil parameters file is offset, the tables must be cut and pasted
such that a single row of blank cells lies between the tables, otherwise
NutsFor will not be able to read the tables. The total number of soil
layers (Nb\_Layers) must also be updated in the General Data File before
running the model.

\textbf{\emph{Depth and Thickness:}}

The depth of the soil layer represents the vertical distance of the soil
layer from the surface of the soil, it is not used in any calculations.
Soil layer thickness determines how thick each soil layer is, it is
recommended that soil layers are set to less than or equal to 20cm
thick. Overly thick soil layers may be liable to over-leaching of
nutrients from the soil layers.

\textbf{\emph{Bulk Density, Solid Density, and Stoniness:}}

Bulk density is defined as fine earth bulk density (particles less than
2mm in diameter). Solid density is not part of the calculations of
NutsFor, and can be ignored. Stoniness is the percent of soil particles
greater than 2mm in diameter, this parameter affects soil moisture and
hydrology calculations and is important to calculate.

\textbf{\emph{Root Fraction and Uptake Fraction per Layer:}}

Keeps trace of the root distribution of the stand between the soil
layers, however it does not determine the uptake of nutrients per layer.
The Uptake Fraction per layer parameter truly determines how the stand
extracts nutrients from the soil layers, allowing for greater
flexibility in determining tree uptake dynamics. Intra-annual uptake
dynamics are changed through the Tree Parameter file.

\textbf{pKGibb, pKAl\_Precip, pCO2:}

The pKGibb parameter is the -log of the reaction constant of gibbsite
dissolution to free aluminum, similarly the pKAl\_precip is the -log of
the reaction constant of Al and its precipitated forms (i.e., the
solubility product). The range of pKAl\_precip (or Ksp of Al(OH)3(s))
should be in the range of 32-35 depending on the pH of the soil layer,
pKGibb ranges from 6-9. These aluminum parameters are important in
determining pore water pH, as Al is the largest contributor of H+
compared to other acids in the model. pCO2 is the partial pressure of
carbon dioxide in the soil solution, it is a constant multiple of
atmospheric CO2 and is not dynamic with decomposition rates.

\textbf{Mineral Area}

This parameter designates the total weatherable mineral surface area of
the soil. It is calculated from the equation:

General Equation 1:

\[
(8*X.Clay)+(2.2*X.Silt)+(0.03*X.Sand)
\]

Citation:

Note, that clays dominant in high area clays may be misrepresented by
this equation \emph{(Citation)}

A more detailed equation, which takes into consideration different sand
and silt size particles;

General Equation 2: \[
()+()+()
\] Citation:

The mineral area parameter is historically difficult to obtain
empirically, clay mineralogy plays a large role in determining this
parameter, and is seldom measured to its full capacity (SOURCE). The
fraction of soil particles in each size class (X.Clay, X.Silt, and
X.Sand) are to be calculated from the whole soil (including coarse
fragment count), not just the fine particulate fraction. Since this
parameter is difficult to calculate, it will be likely that simulated
mineral weathering rates are lower than those observed or calculated for
a given stand. In this case the mineral weathering rates must be
calibrated by increasing (or decreasing, if weathering is too high) the
mineral area parameter. Mineral weathering rates are linked to proton
concentrations, lower pH stimulates mineral weathering. Mineral
weathering thus consumes protons and can be included in the calibration
of pH and alkalinity. Thus, another way of calibrating mineral
weathering rates can be the calibration of proton concentrations.

Soil Moisture Parameters

\begin{enumerate}
\def\labelenumi{\arabic{enumi}.}
\item
  Wilting Point

  The percent soil moisture which corresponds to the wilting point;
  i.e., the pressure of water in the soil matrix is at -15kPa. Wilting
  point may have a different reference pressure depending on the texture
  of the soil.
\item
  Field Capacity

  The percent soil moisture which corresponds to how much water the soil
  can hold after drainage of excess water. Typically this measurement is
  set to the reference pressure of -33kPa.
\item
  Saturation

  The percent of soil moisture corresponding to all pore space in soil
  being filled with water. It can be set as the effective porosity of
  the soil.
\end{enumerate}

CEC Parameters and CEC Initial

\begin{enumerate}
\def\labelenumi{\arabic{enumi}.}
\item
  CEC

  This parameter is the effective cation exchange capacity (ECEC) of
  each soil layer. This parameter is usually measured for every soil and
  soil layer for every depth increment, however there are multiple
  methods through which this can be accomplished (Citation, General
  technical manual). In general, ECEC can be calculated as:

  General Equation \#X.1 \[
  ECEC= 2[Ca^{+2}]+ 2[Mg^{+2}]+[K^+]+[Na^+]+3[Al^{+3}]+[NH_{4}^{+}]
  \]
\item
  Gapon Selectivity Coefficients (KH\_Cation)

  These coefficients determine the propensity of each cation to be
  replaced on the CEC with H+. High selectivity coefficients indicate
  that a cation has a high affinity for the CEC, and are thus liable to
  stay on the CEC rather than weather off due to the presence of other
  cations. Low coefficients will likely lead to high CEC weathering of
  cations, which may result in high soil solution pH. This is due to
  both the consumption of protons in the soil solution, which kick off
  low-affinity cations, as well as the increase in acid neutralizing
  capacity (ANC) that comes with increased base cation concentrations.
  This is because ANC is calculated as the sum of base cations minus the
  sum of acid anions. When ANC is near or above 0, soil solution pH will
  be difficult to constrain.

  Gapon coefficients are to be calculated as:

  General Equation \#X.2 \[
  K_{H:Cation^{+n}}= (E_{Cation^{+n}}/[Cation^{+n}])*([H^{+}]/E_{H})^{n}
  \] Where; \[
  E_{H}=[H^{+}]_{adsorbed}/ECEC
  \] and \[
  E_{Cation^{+n}}=[Cation^{+n}]_{adsorbed}/ECEC
  \] Note that \[
  n = Valence
  \] All concentration units are in mmol/L. It should be noted that
  these values heavily rely on measured initial cation concentrations,
  and so they may be manipulated to get a better match between adsorbed
  and dissolved cations. Note that increasing the selectivity
  coefficient of one cation will change the relative selectivity of all
  cations. For instance, increasing Ca selectivity will lead to more Ca
  adsorption to the CEC, which will in turn kick off other cations,
  assuming the selectivity of the other cations remains the same.
\item
  CEC Initial

  This parameter is not under the ``CEC parameters'' heading in the Soil
  Parameter file, rather it is located at the bottom of the Soil
  Parameter File (it it the last table, named CEC Initial). This
  parameter determines the total adsorbed supply of cations initially on
  the cation exchange complex.

  \hypertarget{aec-parameters}{%
  \subsubsection{AEC Parameters}\label{aec-parameters}}

  Anion Exchange Capacity (AEC) is implemented in the model for 3
  anions; Sulfate, Phosphate, and Chloride. Sulfate adsorption can be
  modeled using either a Langmuir or Freundlich adsorption isotherm.
  Chloride is modeled using the Langmuir model, Phosphate the Freundlich
  model. Sulfate adsoprtion isotherms can be selected in the General
  Data File (Input Table 4).

  \begin{enumerate}
  \def\labelenumii{\arabic{enumii}.}
  \item
    Soil Adsorbed Anions Parameters

    Each anion has two AEC adsorption parameters associated with it,
    Parameter \#1 and Parameter \#2. These parameters will stand for
    different variables depending on the adsoprtion isotherm selected.

    Langmuir

    Freundlich
  \item
    Soil Adsorbed Anions Initial
  \end{enumerate}
\end{enumerate}

\begin{Shaded}
\begin{Highlighting}[]
\NormalTok{Parameter\_T3\textless{}{-}}\KeywordTok{c}\NormalTok{(}\StringTok{"Litter Concentration (N, Ca, Mg,K, S, P)"}\NormalTok{, }\StringTok{"Litter\textasciigrave{}\_C\textasciigrave{}\_Pool"}\NormalTok{, }\StringTok{"Layer\textasciigrave{}\_frac (Fine, Coarse, Humus)"}\NormalTok{, }\StringTok{"Decomp\textasciigrave{}\_rate"}\NormalTok{, }\StringTok{"CO2FACT"}\NormalTok{, }\StringTok{"Nutrient\textasciigrave{}\_Factor (DOCFACT, N\textasciigrave{}\_Fact, Ca\textasciigrave{}\_Fact, Mg\textasciigrave{}\_Fact, K\textasciigrave{}\_Fact, S\textasciigrave{}\_Fact, P\textasciigrave{}\_Fact"}\NormalTok{)}

\NormalTok{Units\_T3\textless{}{-}}\KeywordTok{c}\NormalTok{(}\StringTok{"mmol Nutrient/mol C"}\NormalTok{, }\StringTok{"$mmol C/m\^{}2$"}\NormalTok{, }\StringTok{"N/A"}\NormalTok{, }\StringTok{"$mmol/(m\^{}2*mo)$"}\NormalTok{, }\StringTok{"N/A"}\NormalTok{, }\StringTok{"mmol Nutrient/mol C (DOCFACT, N\textasciigrave{}\_Fact, Ca\textasciigrave{}\_Fact, Mg\textasciigrave{}\_Fact, K\textasciigrave{}\_Fact, S\textasciigrave{}\_Fact, P\textasciigrave{}\_Fact)."}\NormalTok{)}

\NormalTok{Description\_T3\textless{}{-}}\KeywordTok{c}\NormalTok{(}\StringTok{"Determines concentration of nutrient elements in the litter fraction of OM."}\NormalTok{, }\StringTok{"Sets the litter carbon pool, and thus the amount of nutrients per unit area."}\NormalTok{, }\StringTok{"Designates the OM distribution between coarse litter (first column), fine litter (second column), and humus (third column)"}\NormalTok{, }\StringTok{"Calibration decomposition factor shows up multiple times in the file for the 3 OM fractions (1{-}3) and for the mineral soil layers (1{-}n)."}\NormalTok{, }\StringTok{"Calibration factor that determines the proportion of decomposed carbon that effluxes as carbon dioxide."}\NormalTok{, }\StringTok{"Calibration factors for both the 3 OM fractions (1{-}3) and the mineral soil fractions (1{-}n). Can be used to calibrate soil solution and liter flow ion fluxes, range from 0 to 1."}\NormalTok{)}

\NormalTok{text\_tbl\_}\DecValTok{3}\NormalTok{ \textless{}{-}}\StringTok{ }\KeywordTok{data.frame}\NormalTok{(}
\NormalTok{Parameter\_T3, Units\_T3, Description\_T3}
\NormalTok{)}

\KeywordTok{kbl}\NormalTok{(text\_tbl\_}\DecValTok{3}\NormalTok{, }\DataTypeTok{caption=}\StringTok{"Om Decomposition File"}\NormalTok{, }\DataTypeTok{booktabs =} \OtherTok{TRUE}\NormalTok{, }\DataTypeTok{col.names=} \KeywordTok{c}\NormalTok{(}\StringTok{"Input Parameter"}\NormalTok{, }\StringTok{"Units"}\NormalTok{, }\StringTok{"Description"}\NormalTok{), }\DataTypeTok{escape=}\OtherTok{FALSE}\NormalTok{) }\OperatorTok{\%\textgreater{}\%}\StringTok{ }\KeywordTok{kable\_styling}\NormalTok{(}\DataTypeTok{full\_width =}\NormalTok{ F, }\DataTypeTok{latex\_options =} \KeywordTok{c}\NormalTok{(}\StringTok{"scale\_down"}\NormalTok{, }\StringTok{"striped"}\NormalTok{, }\StringTok{"hold\_position"}\NormalTok{))}
\end{Highlighting}
\end{Shaded}

\begin{table}[!h]

\caption{\label{tab:unnamed-chunk-5}Om Decomposition File}
\centering
\resizebox{\linewidth}{!}{
\begin{tabular}[t]{lll}
\toprule
Input Parameter & Units & Description\\
\midrule
\cellcolor{gray!6}{Litter Concentration (N, Ca, Mg,K, S, P)} & \cellcolor{gray!6}{mmol Nutrient/mol C} & \cellcolor{gray!6}{Determines concentration of nutrient elements in the litter fraction of OM.}\\
Litter`_C`_Pool & $mmol C/m^2$ & Sets the litter carbon pool, and thus the amount of nutrients per unit area.\\
\cellcolor{gray!6}{Layer`_frac (Fine, Coarse, Humus)} & \cellcolor{gray!6}{N/A} & \cellcolor{gray!6}{Designates the OM distribution between coarse litter (first column), fine litter (second column), and humus (third column)}\\
Decomp`_rate & $mmol/(m^2*mo)$ & Calibration decomposition factor shows up multiple times in the file for the 3 OM fractions (1-3) and for the mineral soil layers (1-n).\\
\cellcolor{gray!6}{CO2FACT} & \cellcolor{gray!6}{N/A} & \cellcolor{gray!6}{Calibration factor that determines the proportion of decomposed carbon that effluxes as carbon dioxide.}\\
\addlinespace
Nutrient`_Factor (DOCFACT, N`_Fact, Ca`_Fact, Mg`_Fact, K`_Fact, S`_Fact, P`_Fact & mmol Nutrient/mol C (DOCFACT, N`_Fact, Ca`_Fact, Mg`_Fact, K`_Fact, S`_Fact, P`_Fact). & Calibration factors for both the 3 OM fractions (1-3) and the mineral soil fractions (1-n). Can be used to calibrate soil solution and liter flow ion fluxes, range from 0 to 1.\\
\bottomrule
\end{tabular}}
\end{table}

Description of Parameters: OM Decomposition

\textbf{Litter Parameters}

Litter parameters can be broken up into 3 sections:

\begin{enumerate}
\def\labelenumi{\arabic{enumi}.}
\tightlist
\item
  Litter Concentration (N, Ca, Mg, K, S, P)
\item
  Litter Pool (Litter\_C\_Pool, Litter\_Fraction)
\item
  Litter Nutrient Release Dynamics
\end{enumerate}

I). Litter Concentration

This table sets the concentration of main nutrient cations in the litter
layer relative to the amount of carbon in the litter.

II). Soil\_Layer:

1). The coarse litter fragment of the litter layer (fresh litter
included)

2). The fine litter fragment

3). The humus (``Very fine SOM'') fraction

Decomp\_Rate: The k of decomposition, determining how much carbon
decomposes from litter pools.

Co2\_Fact: Determines the fraction of decomposed carbon which is evolved
as carbon dioxide

N\_Fact, Ca\_Fact, Mg\_Fact, K\_Fact, S\_Fact, P\_Fact: Determines
release rate of corresponding nutrient relative to C release due to
decomposition. These parameters can be tweaked to globally slow the
mobilization of nutrients from the litter pool.

DOCFACT:

This factor determines the fraction of the decomposed C pool at each
time step and for each litter pool which is released as DOC. Increasing
this factor increases DOC and thus organic acid (R-) concentrations in
the simulated soil. This tends to lower pH according to the set pKa of
organic acids (See OA\_pKa in Input table 4).

DOC\_Decomp:

Determines the proportion of DOC evolved from decomposing litter that is
itself decomposed. This parameter decreases OA (R-), however it does not
link back to carbon dioxide release as carbon dioxide concentration is a
fixed parameter (See, Input table 2, pCO2)

\begin{longtable}[]{@{}l@{}}
\toprule
Input Table 4: General Data File\tabularnewline
\midrule
\endhead
Parameter Name in File\tabularnewline
Start\_year\tabularnewline
End\_Year\tabularnewline
Nb\_Layers\tabularnewline
Nb\_Minerals\tabularnewline
Mineral\_names\tabularnewline
OA\_pka\tabularnewline
Aluminum\tabularnewline
Phosphoric\_acid\tabularnewline
DIC\tabularnewline
Organic\_acid\_size\tabularnewline
Nb\_Tree\_Comp\tabularnewline
\bottomrule
\end{longtable}

Description of Parameters: General Data File

\begin{longtable}[]{@{}l@{}}
\toprule
Input Table 5: Mineral Data File\tabularnewline
\midrule
\endhead
Parameter Name in File\tabularnewline
Mineral\tabularnewline
pKH, pKH2O, pKCO2, pKr\tabularnewline
kH, kH20, kCO2,kr\tabularnewline
nH, nCO, nr\tabularnewline
WAlH, WBCH, WAlH2O, WBCH2O\tabularnewline
CAl\emph{10\^{}-6, CR}10\^{}-6 CBC*10\^{}-6\tabularnewline
Cation Composition of Minerals (Ca, Mg, K, Na, Al, Si,
PO4)\tabularnewline
Iso\tabularnewline
\bottomrule
\end{longtable}

Description of Parameters: Mineral Data File

\begin{longtable}[]{@{}l@{}}
\toprule
Input Table 6: Tree Parameter File\tabularnewline
\midrule
\endhead
Parameter Name in File\tabularnewline
\bottomrule
\end{longtable}

Description of Parameters: Tree Parameter File

Output File Description

Calibration

Hypothesis Testing

\hypertarget{list-of-equations}{%
\subsection{List of Equations}\label{list-of-equations}}

\textbf{NutsFor Encoded Equations}

\textbf{General Equations}

\hypertarget{literature-cited}{%
\subsection{Literature Cited}\label{literature-cited}}

Appendix, maybe include screenshots of input interface?

\end{document}
